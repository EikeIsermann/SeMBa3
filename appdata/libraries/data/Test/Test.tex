%\noembedding



\title{Project Mobile - \\Eine P2P-Anwendung zum Thema Inklusion}
\subtitle{Entwurf, Realisierung und Evaluation von Prototypen\\(Arbeitstitel)}
\author[David Cyborra]{David Cyborra} 
\institute{Informatik IX, Mensch-Computer-Interaktion\\
Universität Würzburg}

\titlepage


% ------------------------------------------------------------ %


\titledsection{Einleitung}
\begin{subsection}{Einleitung}


\begin{frame}{Was ist ''Project Mobile''?}
\begin{definition}
	Entstehen soll eine \textit{peer-to-peer} Anwendung für Desktop-PCs und mobile Endgeräte, welche Menschen mit Behinderung und solche ohne zusammenbringen soll. Kernpunkt ist das \strong{Zeichen 'B' im Schwerbehindertenausweis}, das einem Begleiter erlaubt kostenlos öffentliche Verkehrsmittel zu nutzen und viele andere Vergünstigungen in Anspruch zu nehmen.\\
\end{definition}
\pause
\strong{$\rightarrow$ Beide Seiten profitieren}
\end{frame}


\end{subsection}


% ------------------------------------------------------------ %


\titledsection{Motivation}
\begin{subsection}{Motivation - Probleme}
\begin{frame}{Motivation, Probleme und Ziele}

\begin{2columns}[0.35] %sizeOfFirstCol=[0.0,1.0]
{\centering\includegraphics[scale=0.34]{figures/own/zachi.png}}
{ 


\pause
\begin{titleditems}{Problemstellung}[<+->]
\pause
	\item[1.]Sehr geringe Barrierefreiheit bei der deutschen Bahn \cite{wittmann}
	\pause
	\begin{itemize}
	  \item 24 Stunden vor Reise ankündigen
	  \item Oftmals nur ein Rollstuhlplatz
	  \item An vielen Bahnhöfen nur wenig Servicepersonal
	  \item Unterschiedliche Bahnsteighöhen
	\end{itemize} 
	\item[2.]Wenig Bewusstsein unter Menschen ohne Behinderung \cite{wittmann}
\end{titleditems}
\pause

}
\end{2columns}

\end{frame}
\end{subsection} %Mot - Probleme


% ------------------------------------------------------------ %


\begin{subsection}{Motivation und Ziele}
\begin{frame}{Motivation, Probleme und Ziele}

\begin{titleditems}{Motivation}[<+->]
\pause
\item Abhilfe bzgl. Barrieren schaffen
\item Die Bahn (und Andere) auf Probleme hinweisen
\item Allgemeines Bewusstsein schaffen
\end{titleditems}

\pause

\begin{titleditems}{Ziele}[<+->]
\pause
\item Deutlich machen, dass auch Menschen mit Behinderung spontan sein wollen
\item Menschen mit Behinderung mobiler machen
\noindent\rule{8cm}{0.4pt}
\item HCI-Projekt und Masterthesis gut abschließen
\item Release der App in diesem Rahmen
\end{titleditems}

\end{frame}
\end{subsection} %Motivation und Ziele


% ------------------------------------------------------------ %


\titledsection{Verwandte Arbeiten}

\begin{subsection}{Soziale Netze}
\begin{frame}{Soziale Netzwerke und Mitfahrgelegenheiten}

\begin{2columns}[0.35] %sizeOfFirstCol=[0.0,1.0]
{\centering\includegraphics[scale=0.1]{figures/own/social_networks.png}}
{ 
\pause
\begin{itemize}
	\item Zwischen einer und 34 Millionen Nutzer \cite{statista:2014}
	\pause
	\item Aber: Keine direkte Konkurrenz
	\pause
	\begin{itemize}
	  \item[$\rightarrow$] Keine 1 zu 1 Relationen als Ziel
	  \item[$\rightarrow$] Kein Fokus auf Events/Unternehmungen
	  \noindent\rule{4cm}{0.4pt}
	  \pause
	  \item[$\rightarrow$] Keine Spezialisierung auf Menschen mit Behinderung
	\end{itemize} 
	\pause
	 $\rightarrow$ \strong{Aber: Hohe Akzeptanz}
\end{itemize}
 }
 
\end{2columns}

\end{frame}
\end{subsection} %Soziale Netze


% ------------------------------------------------------------ %


\begin{subsection}{Inklusion}
\begin{frame}{Inklusion von Menschen mit Behinderung}

\begin{titleditems}{1. Nicht-interaktive Informationsdienste:}
\pause
\item Wheelmap.org \cite{wheelmap}
\item Inklusionslandkarte.de \cite{Inklusionslandkarte}
\item Handicapnet.com \cite{handicapnet}
\item viele mehr..
\end{titleditems}

\begin{itemize}[<+->]
\pause
\item[\strong{$\rightarrow$}]Hauptsächlich Rollstuhlfahrer als Zielgruppe
\item[\strong{$\rightarrow$}]Rein informativ \pause \strong{$\rightarrow$ keine direkte Absprache}
\end{itemize}

\end{frame}

%----------------------------------

\begin{frame}{Inklusion von Menschen mit Behinderung}


\begin{titleditems}{2. Dienste / Ansätze mit gleichem Ziel:}
\pause
\item \strong{''Ziemlich Beste Begleiter''} \cite{ziemlichbestebegleiter}
	\pause
	\begin{itemize}[<+->]
	\item Inserate \pause \strong{$\rightarrow$ keine direkte Absprache} \pause
	\item Kaum benutzt (Letzter Eintrag August '14)
	\item Nicht gut bedienbar
	\item Nicht wirklich responsiv (Landscape)
	\end{itemize}
	\pause
\item \strong{''Kulturbörse''} \cite{dieglocke}
	\pause
	\begin{itemize}
	\item Bisher nur Idee im Rahmen der Aktion ''Inklusion 2016''
	\item Lokales Projekt in Augsburg
	\end{itemize}

\end{titleditems}

\pause
\strong{\strong{$\rightarrow$}Aber: Interesse wächst}


\end{frame}

\end{subsection} %Inklusion


% ------------------------------------------------------------ %


\titledsection{Eigener Ansatz}


\begin{subsection}{Kernpunkte}
\begin{frame}{Kernpunkte des eigenen Ansatzes}

\strong{Schlussfolgerungen aus verwandten Arbeiten:}
\pause
\begin{itemize}[<+->]
\item[1.]Usability und User Experience\\
\item[2.]Accessibility\\
\item[3.]WebApp \pause \strong{$\rightarrow$ Plattformübergreifende Applikation}
\end{itemize}

\end{frame}
\end{subsection} %Kernpunkte


% ------------------------------------------------------------ %


\begin{subsection}{Usability und User Experience}
\begin{frame}{Usability und User Experience}

\pause

\begin{question}[Frage]
Wie erreichen?
\end{question}

\pause

\begin{itemize}[<+->]
\item Erfahrung im User Centered Design
\item Vorstudie zur Bestimmung der Zielgruppe
\item Iterative Softwareentwicklung - \pause \strong{Papierprototyp:}
	\pause
	\begin{itemize}
	\item Evaluation durch 4 Experten
	\item[\strong{$\rightarrow$}]Qualitative Ergebnisse (Usability)
	\end{itemize}
\item Iterative Softwareentwicklung - \pause \strong{Virtueller Prototyp:}
	\pause
	\begin{itemize}
	\item Aus Resultaten der Expertenbefragung
	\item Novizenstudie mit n $\sim$ 30
	\item[\strong{$\rightarrow$}]Quantitative Ergebnisse (auch UX)
	\end{itemize}
\end{itemize}

\end{frame}
\end{subsection} %Usab und UX


% ------------------------------------------------------------ %


\begin{subsection}{Accessibility}
\begin{frame}{Accessibility}

\pause

\begin{question}[Frage]
Wie erreichen?
\end{question}

\pause

\begin{itemize}[<+->]

\item Einfachste Bedienung
\item Hoher Kontrast $\rightarrow$ korrekte Farbgestaltung
\item Typografie
\item Bedienelemente in passender Größe/skalierbar
\item Screenreader

\noindent\rule{\linewidth}{0.4pt}
\pause

\item Einfache Sprache?

\end{itemize}

\end{frame}
\end{subsection} %Accessibility


% ------------------------------------------------------------ %


\titledsection{Aktueller Stand}


\begin{subsection}{Vorstudie}
\begin{frame}{Vorstudie}


\begin{titleditems}{1. Aufbau:}[<+->]
\pause
\item Limesurvey \cite{limesurvey}
\item 25 Fragen in 5 Gruppen $\rightarrow$ ca. 10 Minuten
\item "2-Rollen-System"
	\pause
	\begin{itemize}
	\item Frage in erster Fragengruppe bestimmt "Rolle"
	\item Weitere Fragen dementsprechend ausgewählt
	\item[$\rightarrow$]Sinnvoller Ablauf
	\item[$\rightarrow$]Einfachere Auswertung
	\end{itemize}
\end{titleditems}

\end{frame}

%-------------------------------

\begin{frame}{Vorstudie}


\begin{titleditems}{2. Ergebnisse (stehen noch aus)}[<+->]
\pause
\item Nutzungskontext
	\pause
	\begin{itemize}
	\item Nutzung von PC \& Laptop
	\item Nutzung von Smartphones \& Tablets
	\item Nutzung des Internet
	\item Usw.
	\end{itemize}
	
\item Nutzerprofil/Personas
	\pause
	\begin{itemize}
	\item Alter
	\item (Grad der) Behinderung
	\item Erfahrung mit Behinderten etc.
	\item Personas
	\end{itemize}
	
\end{titleditems}

\end{frame}

\end{subsection} %Vorstudie


% ------------------------------------------------------------ %


\begin{subsection}{Papierprototyp}

\begin{frame}{Papierprototyp}

\strong{Grundidee:\\}
Schnelle Sketches \pause $\rightarrow$ akkurate  Views \pause $\rightarrow$ Prototyp\\

\pause

\begin{question}[Frage]
Warum ein Papierprototyp?
\end{question}

\end{frame}

%--------------------------------------

\begin{frame}{Papierprototyp}

\begin{titleditems}{Vorteile Papierprototyp: \cite{Arnowitz:2006:EPS:1196696}}[<+->] 
\pause
\item[\strong{+}]Hohe Fidelität (''Degree of detail and finish'')
\item[\strong{+}]Änderungen können sofort vorgenommen werden
\item[\strong{+}]Geringer Aufwand
\item[\strong{+}]Keine systembezogenen Einflüsse
\item[\strong{+}]Unterstützung iterativer Entwicklung (mehr Ideen)
\end{titleditems}


\end{frame}
\end{subsection} %Papierprototyp


% ------------------------------------------------------------ %


\begin{subsection}{Experten Evaluation}
\begin{frame}{Interner Review und Experten Evaluation}

\begin{titleditems}{Evaluation des Papierprototypen:}
\pause
\item Interner Review nach unabhängiger Entwicklung\\
\pause

\item Evaluation mit 4 Experten:\\
\pause
	\begin{itemize}
	\item Abschluss MCS und Studium der HCI
	\item Cognitive Walkthrough
	\item 3 Use-Cases (restliche Views auch besehen)
	\item Thinking-Aloud und Sketchen
	\end{itemize}
\end{titleditems}



\end{frame}
\end{subsection} %Experten Evaluation

\begin{subsection}{Ergebnisse}

\begin{frame}{Ergebnisse der Evaluation}

\begin{titleditems}{Was lernen wir?}[<+->]
\pause
\item \underline{Use cases sind Pflicht!}
\item Viele kleinere Anmerkungen
\item Einige (wenige) must-Anmerkungen
	\begin{itemize}
	\item[\strong{$\rightarrow$}]Fast alle von allen Experten gleich
	\end{itemize}
\end{titleditems}

\end{frame}

%------------------------------

\begin{frame}{Beispiel: Prototyp - Nachrichten}

\centering\includegraphics[scale=0.3]{figures/own/chat_ohne.jpg}

\end{frame}

%------------------------------

\begin{frame}{Beispiel: Prototyp - Nachrichten ergänzt}

\centering\includegraphics[scale=0.3]{figures/own/chat_mit.jpg}

\end{frame}

\end{subsection} %Ergebnisse der Experten Evaluation


% ------------------------------------------------------------ %


\titledsection{Ausblick}


\begin{subsection}{HCI-Projekt}
\begin{frame}{Im Rahmen des HCI-Projekts}


	\begin{titleditems}{Nächste Schritte:}[<+->]
	\pause
	\item[1.]Auswertung Vorstudie
	\item[2.]Digitaler interaktiver (virtueller) Prototyp \cite{Arnowitz:2006:EPS:1196696}
		\pause
		\begin{itemize}
		\item[\strong{-}]Geringe Fidelität
		\item[\strong{+}]Geringe Kosten
		\item[\strong{+}]Geringer Aufwand
		\item[\strong{+}]Kontext und Verhalten wie Endprodukt
		\end{itemize}

	\pause

	\item[3.]Novizenstudie
		\pause
		\begin{itemize}
		\item $\sim$ 30 Probanden
		\item Menschen mit relevanten Behinderungen (B) und gesunde
		\noindent\rule{8cm}{0.4pt}
		\item 3 Use cases
		\item Videoaufzeichnung mit Kommentaren
		\item (Eigenes Sketchen)
		\item Abschluss mit AttrakDiff/QUESI/Nasa TLX
		\end{itemize}
	\end{titleditems}

\end{frame}
\end{subsection} %HCI-Projekt


% ------------------------------------------------------------ %


\begin{subsection}{Masterthesis}
\begin{frame}{Im Rahmen der Masterthesis}



	\begin{titleditems}{Ab März:}[<+->]
		\item[4.]Redesign aus Ergebnissen der Novizenstudie
		\item[5.]\strong{Implementierung}
		\item[6.]Finale Evaluation zur Vermeidung von Fehlern bei Release
		\item[7.]Go-Live 
	\end{titleditems}
	
\end{frame}
\end{subsection} %Masterthesis

\begin{frame}
	\centering{\strong{Anregungen bitte!}}
\end{frame}

% ------------------------------------------------------------ %


\bibliographystyle{abbrv}
\bibliography{bibliography}
